\chapter{Аналитический раздел}
\label{cha:analysis}

В данном разделе будет определена теоретическая база, необходимая для реализации поставленных задач.

\section{Описание алгоритмов}

Данные алгоритмы основываются на применении формул Левенштейна и Дамерау-Левенштейна. Рассмотрим эти формулы подробнее.

\subsection{Расстояния Левенштейна}

Расстояние Левенштейна между двумя строками - это минимальная сумма произведений количества операций вставки, удаления и замены одного символа, необходимых для превращения одной строки в другую, на их стоимость.

Вышеописанные операции имеют следующие обозначения:
\begin{itemize}
    \item $I$ ($insert$) - вставка;
    \item $D$ ($delete$) - удаление;
    \item $R$ ($replace$) - замена;
\end{itemize}

При этом $cost(x)$ есть обозначение стоимости некоторой операции x. Будем считать, что символы в строках нумеруются с первого. Пусть $S_{1}$ и $S_{2}$ - две строки с длинами N и M соответственно. Тогда расстояние Левенштейна D(M, N) вычисляется по формуле (1.1):
\begin{equation}
D(i,j) = \left\{ \begin{array}{ll}
 0, & \textrm{$i = 0, j = 0$}\\
 i * cost(D), & \textrm{$j = 0, i > 0$}\\
 j * cost(I), & \textrm{$i = 0, j > 0$}\\
min(\\
D(i,j-1) + cost(I),\\
D(i-1, j) + cost(D), &\textrm{$j > 0, i > 0$}\\
D(i-1, j-1) + mrcost(S_1[i], S_2[j])\\
)
  \end{array} \right.
\end{equation}
где $min(a, b, c)$ возвращает наименьшее значение из $a, b, c$; а $mrcost(x_1, x_2)$ - 0, если символы $x_1, x_2$ совпадают, и $cost(R)$ иначе.

\subsection{Расстояние Дамерау-Левенштейна}
Определение расстояния Дамерау-Левенштейна аналогично определению расстояния Левенштейна с учётом новой операции - перестановки соседних символов (транспозиции). Соответственно, обозначения операций:
\begin{itemize}
    \item $I$ ($insert$) - вставка;
    \item $D$ ($delete$) - удаление;
    \item $R$ ($replace$) - замена;
    \item $T$ ($transpose$) - перестановка соседних символов.
\end{itemize}

При тех же обозначениях имеем формулы (1.2) и (1.3):
\begin{equation}
D(i,j) = \left\{ \begin{array}{ll}
 min(A, D(i - 2, j - 2) + cost(T), & \textrm{$i > 1, j > 1,$}\\
 & \textrm{$S_1[i] = S_2[j - 1],$}\\
 & \textrm{$S_1[i - 1] = S_2[j]$}\\
 A & \textrm{Иначе}\\
  \end{array} \right.
\end{equation}
где A:
\begin{equation}
A = \left\{ \begin{array}{ll}
 0, & \textrm{$i = 0, j = 0$}\\
 i * cost(D), & \textrm{$j = 0, i > 0$}\\
 j * cost(I), & \textrm{$i = 0, j > 0$}\\
min(\\
D(i,j-1) + cost(I),\\
D(i-1, j) + cost(D), &\textrm{$j > 0, i > 0$}\\
D(i-1, j-1) + mrcost(S_1[i], S_2[j])\\
)
  \end{array} \right.
\end{equation}

\section{Вывод}
Очевидно, формулы Левенштейна и Дамерау-Левенштейна имеют различную прикладную направленность. Если вторая рассчитана больше на слова, набранные человеком, то первая - нет, так как транспозиция фактически не является тривиальной операцией, и её наличие это условность, требуемая контекстом применения.

