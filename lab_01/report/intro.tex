\Introduction

Целью данной работы является изучение динамического программирования на материале алгоритмов Левенштейна и Дамерау-Левенштейна.

Данные алгоритмы решают проблему поиска редакционного расстояния между двумя строками. Редакционное расстояние определяется количеством некоторых операций, необходимых для превращения одного слова в другое, а так же стоимостью этих операций.

Для достижения поставленной цели необходимо решить следующие задачи:

\begin{itemize}
    \item изучение алгоритмов Левенштейна и Дамерау-Левенштейна нахождения расстояния между строками;
    \item применение метода динамического программирования для матричной реализации указанных алгоритмов;
    \item получение практических навыков реализации указанных алгоритмов: двух алгоритмов в матричной версии и одного из алгоритмов в рекурсивной версии;
    \item сравнительный анализ линейной и рекурсивной реализаций выбранного алгоритма определения расстояния между строками по затрачиваемым ресурсам (времени и памяти);
    \item экспериментальное подтверждение различий во временнóй эффективности рекурсивной и нерекурсивной реализаций выбранного алгоритма определения расстояния между строками при помощи разработанного программного обеспечения на материале замеров процессорного времени выполнения реализации на варьирующихся длинах строк;
    \item описание и обоснование полученных результатов в отчете о выполненной лабораторной работе, выполненного как расчётно-пояснительная записка к работе.
\end{itemize}

