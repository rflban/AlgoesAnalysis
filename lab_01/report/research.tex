\chapter{Исследовательский раздел}
\label{cha:research}

В данном разделе будет продемонстрирована и проанализирована работа разработанной программы поиска редакционного расстояния на основе формул Левенштейна и Дамерау-Левенштейна.

\section{Примеры работы}
Рассмотрим примеры работы программы. На рисунках 4.1, 4.2, 4.3, 4.4 представлены случаи с двумя переставленными соседними буквами, пропущенной буквы, пустыми словами, и полностью разными словами соответственно.

\begin{figure}[H]
    \centering
    \includegraphics[scale=0.5]{./pics/t1.png}
    \caption{Транспозиция}
\end{figure}
\begin{figure}[H]
    \centering
    \includegraphics[scale=0.5]{./pics/t2.png}
    \caption{Пропуск одной буквы}
\end{figure}
\begin{figure}[H]
    \centering
    \includegraphics[scale=0.5]{./pics/t3.png}
    \caption{Пустые слова}
\end{figure}
\begin{figure}[H]
    \centering
    \includegraphics[scale=0.5]{./pics/t4.png}
    \caption{Совершенно разные слова}
\end{figure}

\section{Результаты тестирования}
Тестирование всех трёх реализаций алгоритмов прошло успешно. Результаты тестов представлены в таблицах 4.1, 4.2, 4.3.

\begin{table}[H]
    \caption{Результаты тестирования алгоритма Вагнера-Фишера}
	\begin{tabular}{|c|c|c|c|c|}
 	\hline
    \No{} & Строка 1 & Строка 2 & \makecell{Расстояние\\Левенштейна} & \makecell{Ожидаемое расстояние\\Левенштейна} \\
 	\hline
 	1 & some & any & 4 & 4\\
 	\hline
 	2 & & nothing & 7 & 7\\
 	\hline
 	3 & & & 0 & 0\\
 	\hline
 	4 & bashrc & bashcr & 2 & 2\\
 	\hline
 	5 & bus & BuS & 2 & 2\\
 	\hline
 	6 & electricity & city & 7 & 7\\
 	\hline
 	7 & powerful & powerless & 4 & 4\\
 	\hline
 	8 & grow & flow & 2 & 2\\
 	\hline
 	9 & rise & rice & 1 & 1\\
 	\hline
    10 & legal & illegal & 2 & 2\\
 	\hline
    11 & same & same & 0 & 0\\
    \hline
	\end{tabular}
\end{table}

\begin{table}[H]
    \caption{Результаты тестирования рекурсивного алгоритма Дамерау-Левенштейна}
	\begin{tabular}{|c|c|c|c|c|}
 	\hline
    \No{} & Строка 1 & Строка 2 & \makecell{Расстояние\\Дамерау-Левенштейна} & \makecell{Ожидаемое расстояние\\Дамерау-Левенштейна} \\
 	\hline
 	1 & some & any & 4 & 4\\
 	\hline
 	2 & & nothing & 7 & 7\\
 	\hline
 	3 & & & 0 & 0\\
 	\hline
 	4 & bashrc & bashcr & 1 & 1\\
 	\hline
 	5 & bus & BuS & 2 & 2\\
 	\hline
 	6 & electricity & city & 7 & 7\\
 	\hline
 	7 & powerful & powerless & 4 & 4\\
 	\hline
 	8 & grow & flow & 2 & 2\\
 	\hline
 	9 & rise & rice & 1 & 1\\
 	\hline
    10 & legal & illegal & 2 & 2\\
 	\hline
    11 & same & same & 0 & 0\\
    \hline
	\end{tabular}
\end{table}

\begin{table}[H]
    \caption{Результаты тестирования рекурсивного алгоритма Дамерау-Левенштейна}
	\begin{tabular}{|c|c|c|c|c|}
 	\hline
    \No{} & Строка 1 & Строка 2 & \makecell{Расстояние\\Дамерау-Левенштейна} & \makecell{Ожидаемое расстояние\\Дамерау-Левенштейна} \\
 	\hline
 	1 & some & any & 4 & 4\\
 	\hline
 	2 & & nothing & 7 & 7\\
 	\hline
 	3 & & & 0 & 0\\
 	\hline
 	4 & bashrc & bashcr & 1 & 1\\
 	\hline
 	5 & bus & BuS & 2 & 2\\
 	\hline
 	6 & electricity & city & 7 & 7\\
 	\hline
 	7 & powerful & powerless & 4 & 4\\
 	\hline
 	8 & grow & flow & 2 & 2\\
 	\hline
 	9 & rise & rice & 1 & 1\\
 	\hline
    10 & legal & illegal & 2 & 2\\
 	\hline
    11 & same & same & 0 & 0\\
    \hline
	\end{tabular}
\end{table}

\section{Эксперименты по замеру времени}
Чтобы подтвердить вывод об оценке сложности алгоритмов поиска редакционного расстояния, проведём эксперименты по замеру времени и построим графики зависимости времени выполнения данных алгоритмов от длины обрабатываемых слов.

\subsection{Эксперимент 1}
На рисунке 4.5 приведён график сравнения алгоритма Вагнера-Фишера и матричного алгоритма Дамерау-Левенштейна. Для этого эксперимента было сгенерировано 100 пар полностью не совпадающих строк с диапазоном длин от 10 до 1000. Как видно, законы изменения времени выполнения этих алгоритмов практически одинаковы и отличаются лишь на некоторый постоянный коэффициент.
\begin{figure}[H]
    \centering
    \begin{tikzpicture}
        \begin{axis}[
            width=0.8*\linewidth,
            xlabel={Длина слова},
            ylabel={Время (нс)},
            grid=major,
            legend pos=north west,
        ]

        \addplot[color=red]
            table[x=len,y=time,col sep=comma]{./data/wagner-fischer-1.csv};
        \addplot[color=blue]
            table[x=len,y=time,col sep=comma]{./data/damerau-levenshtein-1.csv};

        \legend{Вагнера-Фишера, Дамерау-Левенштейна}

        \end{axis}
    \end{tikzpicture}
    \caption{График сравнения алгоритма Вагнера-Фишера и матричного алгоритма Дамерау-Левенштейна}
\end{figure}

\subsection{Эксперимент 2}
На рисунке 4.6 приведён график сравнения рекурсивного и матричного алгоритмов нахождения расстояния Дамерау-Левенштейна. Для этого эксперимента было сгенерировано 10 пар полностью различных слов с диапазоном длин от 1 до 10. Количество времени, необходимого для выполнения рекурсивного алгоритма, растёт экспоненциально, в то время как сложность матричного алгоритма имеет квадратичный рост, что наглядно изображено на рисунке 4.5.
\begin{figure}[H]
    \centering
    \begin{tikzpicture}
        \begin{axis}[
            width=0.8*\linewidth,
            xlabel={Длина слова},
            ylabel={Время (нс)},
            grid=major,
            legend pos=north west,
        ]

        \addplot[color=red]
            table[x=len,y=time,col sep=comma]{./data/damerau-levenshtein-rec-2.csv};
        \addplot[color=blue]
            table[x=len,y=time,col sep=comma]{./data/damerau-levenshtein-2.csv};

        \legend{рекурсивный, матричный}

        \end{axis}
    \end{tikzpicture}
    \caption{График сравнения рекурсивного и матричного алгоритмов Дамерау-Левенштейна}
\end{figure}

\pagebreak
\section{Вывод}
Как итог, была подтверждена корректная работоспособность реализованной программы нахождения расстояний Левенштейна и Дамерау-Левенштейна и доказаны тезисы, составленные в результате анализа этих алгоритмов.

