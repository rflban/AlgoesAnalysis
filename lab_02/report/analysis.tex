\chapter{Аналитический раздел}
\label{cha:analysis}

В данном разделе будет определена теоретическая база, необходимая для реализации поставленных задач.

В соответствии с книгой \cite{matrixintro} дадим определение произведения двух матриц. Пусть даны прямоугольные матрицы $A$ и $B$. Размеры этих матриц $n\times{}r$ и $r\times{}m$ соответственно. Тогда результатом умножения матрицы $A$ на матрицу $B$ называется такая матрица $C$ размера $n\times{}m$, что:
\begin{equation}
    c_{ij} = \sum\limits_{k=1}^r(a_{ik}\cdot{}b_{kj})
\end{equation}
где $i = \overline{1, n}$, $j = \overline{1, m}$.

\section{Описание алгоритмов}
Рассмотрим алгоритмы вычисления произведения матриц.

\subsection{Классический алгоритм умножения}
Классический алгоритм умножения матриц дословно повторяет определение данной операции:
\begin{equation}
    \begin{bmatrix}
        a_{1 1} & ... & a_{1 r} \\
        ... & ... & ... \\
        a_{n 1} & ... & a_{n r} \\
    \end{bmatrix}
    \times{}
    \begin{bmatrix}
        b_{1 1} & ... & b_{1 m} \\
        ... & ... & ... \\
        b_{r 1} & ... & b_{r m} \\
    \end{bmatrix}
    =
    \begin{bmatrix}
        c_{1 1} & ... & c_{1 m} \\
        ... & ... & ... \\
        c_{n 1} & ... & c_{n m} \\
    \end{bmatrix}
\end{equation}
где $c_{ij}$ вычисляется по формуле (1.1).

\subsection{Алгоритм Винограда}
В качестве примера рассмотрим два вектора длинной 4: $U = (u_1, u_2, u_3, u_4)$ и $V = (v_1, v_2, v_3, v_4)$. Их скалярное произведение равно:
\begin{equation}
    U\times{}V = u_1\cdot{}v_1 + u_2\cdot{}v_2 + u_3\cdot{}v_3 + u_3\cdot{}v_4
\end{equation}
Левую часть равенства (1.3) можно записать в виде:
\begin{equation}
    (u_1 + v_2)\cdot{}(u_2 + v_1) + (u_3 + v_4)\cdot{}(u_4 + v_3) - u_1\cdot{}u_2 - u_3\cdot{}u_4 - v_1\cdot{}v_2 - v_3\cdot{}v_4
\end{equation}
Выражение (1.4) допускает предварительную обработку в случае умножения двух матриц. Для каждой строки можно вычислить выражение (1.5):
\begin{equation}
    u_1\cdot{}u_2 + u_3\cdot{}u_4
\end{equation}
А для каждого столбца - выражение (1.6):
\begin{equation}
    v_1\cdot{}v_2 + v_3\cdot{}v_4
\end{equation}
Таким образом, выходит, что над заранее обработанными данными необходимо выполнить лишь 2 умножения, 5 сложений и 2 вычитания.

\section{Вывод}
Были рассмотрены классический алгоритм алгоритм умножения матриц и алгоритм Винограда, определено, что суть второго заключается в уменьшении количества операций умножения за счёт усложнения последовательности вычислений.

