\chapter{Исследовательский раздел}
\label{cha:research}

В данном разделе будут производиться эксперементы над корректно реализованной программой.

\section{Эксперименты по замеру времени}
На рисунках 4.1 и 4.2 изображены графики зависимости времени выполнения программы от размера входных матриц. Первый график соответствует квадратным матрицам с чётным размером, а второй - нечётным.

\begin{figure}[H]
    \centering
    \begin{tikzpicture}
        \begin{axis}[
            width=0.8*\linewidth,
            xlabel={Длина матрицы},
            ylabel={Время (нс)},
            grid=major,
            xtick={0, 100, ..., 1000},
            legend pos=north west,
        ]

        \addplot[color=red]
            table[x=len,y=time,col sep=comma]{./data/test_even_classic.csv};
        \addplot[color=blue]
            table[x=len,y=time,col sep=comma]{./data/test_even_winograd.csv};
        \addplot[color=green]
            table[x=len,y=time,col sep=comma]{./data/test_even_owinograd.csv};

        \legend{Стандартный алгоритм, Алгоритм Винограда, Модифицированные алгоритма Винограда}

        \end{axis}
    \end{tikzpicture}
    \caption{Чётные размеры}
\end{figure}

\begin{figure}[H]
    \centering
    \begin{tikzpicture}
        \begin{axis}[
            width=0.8*\linewidth,
            xlabel={Длина матрицы},
            ylabel={Время (нс)},
            grid=major,
            xtick={0, 100, ..., 1000},
            legend pos=north west,
        ]

        \addplot[color=red]
            table[x=len,y=time,col sep=comma]{./data/test_uneven_classic.csv};
        \addplot[color=blue]
            table[x=len,y=time,col sep=comma]{./data/test_uneven_winograd.csv};
        \addplot[color=green]
            table[x=len,y=time,col sep=comma]{./data/test_uneven_owinograd.csv};

        \legend{Стандартный алгоритм, Алгоритм Винограда, Модифицированные алгоритма Винограда}

        \end{axis}
    \end{tikzpicture}
    \caption{Нечётные размеры}
\end{figure}

\pagebreak
\section{Вывод}
В результате эксперементов по замеру времени подтвердились тезисы, сделанные в конструкторском разделе: модифицированный алгоритм Винограда работает быстрее алгоритма Винограда, а алгоритм Винограда - быстрее стандартного алгоритма умножения матриц.

