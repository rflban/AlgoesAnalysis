\chapter{Исследовательский раздел}
\label{cha:research}

В данном разделе будут производиться эксперементы над корректно реализованной программой.

\section{Эксперименты по замеру времени}
На рисунках 4.1 и 4.2 изображены графики зависимости времени выполнения программы от размера входных матриц. Первый график соответствует квадратным матрицам с чётным размером, а второй - нечётным.

\begin{figure}[H]
    \begin{tikzpicture}
        \begin{axis}[
            legend pos = north west,
            xlabel=Размер матрицы,
            ylabel=микросекунды,
            grid = major,
            width = 0.8\paperwidth,
            height = 0.38\paperheight,
            line width = 1
        ]
            \legend{
                Стандартный алгоритм,
                Алгоритм Винограда,
                Оптимизированный алгоритм Винограда
            };
            \addplot[dashed] coordinates {
                (100, 34501)
                (200, 273438)
                (300, 911536)
                (400, 2287721)
                (500, 4467987)
                (600, 7982696)
                (700, 12589205)
                (800, 19543501)
                (900, 29907202)
                (1000, 39608083)
            };
            \addplot[black] coordinates {
                (100, 18468)
                (200, 141549)
                (300, 509012)
                (400, 1465966)
                (500, 2716895)
                (600, 4893372)
                (700, 7788205)
                (800, 11974361)
                (900, 17429652)
                (1000, 24443640)
            };
            \addplot[dotted] coordinates {
                (100, 15781)
                (200, 117581)
                (300, 438752)
                (400, 1146582)
                (500, 2466496)
                (600, 4238264)
                (700, 6635427)
                (800, 10541381)
                (900, 15063388)
                (1000, 21113694)
            };
        \end{axis}
    \end{tikzpicture}
    \caption{Чётные размеры}
\end{figure}

\begin{figure}[H]
    \begin{tikzpicture}
        \begin{axis}[
            legend pos = north west,
            xlabel=Размер матрицы,
            ylabel=микросекунды,
            grid = major,
            width = 0.8\paperwidth,
            height = 0.38\paperheight,
            line width = 1
        ]
            \legend{
                Стандартный алгоритм,
                Алгоритм Винограда,
                Оптимизированный алгоритм Винограда
            };
            \addplot[dashed] coordinates {
                (101, 34170)
                (201, 263159)
                (301, 909942)
                (401, 2262964)
                (501, 4486670)
                (601, 7851527)
                (701, 12517730)
                (801, 19429387)
                (901, 38556411)
                (1001, 62124826)
            };
            \addplot[black] coordinates {
                (101, 19163)
                (201, 142118)
                (301, 544590)
                (401, 1417500)
                (501, 2850727)
                (601, 4809657)
                (701, 7629622)
                (801, 12085389)
                (901, 17761017)
                (1001, 24574933)
            };
            \addplot[dotted] coordinates {
                (101, 15982)
                (201, 119376)
                (301, 442958)
                (401, 1257460)
                (501, 2437730)
                (601, 4266685)
                (701, 6876743)
                (801, 10731128)
                (901, 15011152)
                (1001, 23672051)
            };
        \end{axis}
    \end{tikzpicture}
    \caption{Нечётные размеры}
\end{figure}


\pagebreak
\section{Вывод}
В результате эксперементов по замеру времени подтвердились тезисы, сделанные в конструкторском разделе: модифицированный алгоритм Винограда работает быстрее алгоритма Винограда, а алгоритм Винограда - быстрее стандартного алгоритма умножения матриц.

