\chapter{Аналитический раздел}
\label{cha:analysis}

В данном разделе будет определена теоретическая база, необходимая для реализации поставленных задач.

\section{Описание алгоритмов}
Сортировка - это процесс упорядочения некоторого множества элементов, на котором определены отношения порядка >, <, >=, <= (по возрастанию или убыванию) \cite{pav}. При выборе алгоритма сортировки необходимо выбрать тот алгоритм, который будет проделывать минимум операций над данными и тем самым максимально быстро получать необходимый результат - отсортированный список.

Трудоемкость алгоритма - это зависимость количества операций от количества данных, с которыми алгоритм работает. Список действий, цена которых считается за 1:

$$
+, -, *, /, \%, =, ==, !=, <, >, <=, >=, [], +=
$$

За последние 70 лет появилось множество алгоритмов сортировок для компьютера\cite{knuth}.

\subsection{Быстрая сортировка}
Описание алгоритма:
\begin{enumerate}
	\item выбирается элемент, называемый опорным;
	\item остальные элементы сравниваются с опорным, на основании сравнения меньшие опорного перемещаются левее него, а большие или равные - правее;
	\item рекурсивно упорядочиваются подмассивы, лежащие слева и справа от опорного элемента.
\end{enumerate}
Задачи о выборе опорного элемента и разбиении массива на подмассивы относительно опорного элемента могут быть решены разными способами, а эффективность их решения напрямую влияет на эффективность сортировки. В данной лабораторной работе будет использоваться разбиение Ломуто, а в качестве опорного элемента будет выбираться последний элемент.

\subsection{Сортировка вставками}
Описание алгоритма:
\begin{enumerate}
    \item выбирается один из элементов входных данных;
	\item выбранный элемент вставляется на нужную позицию в уже отсортированной последовательности;
	\item п 1,2 выполняются, пока набор входных данных не будет исчерпан.
\end{enumerate}

\subsection{Шейкерная сортировка}
Данный алгоритм является разновидностью сортировки простыми обменами, суть которой заключается в повторяющихся до полной сортированности проходах по по сортируемому массиву, за каждый из которых элементы попарно сравниваются и меняются местами, если их порядок неверный.

Сортировка простыми обменами обладает следующими особенностями:
\begin{itemize}
    \item если при движении по части массива перестановки не происходят, то эта часть массива уже отсортирована и, следовательно, её можно исключить из рассмотрения;
    \item при движении от конца массива к началу элемент минимальной ценности переходит на первую позицию, а наиболее ценный элемент сдвигается только на одну позицию вправо.
\end{itemize}

Алгоритм шейкреной сортировки предлагает учитывать вышеописанные особенности и вносит следующие изменения:
\begin{itemize}
    \item границы рабочей части массива устанавливаются в месте последнего обмена на каждой итерации;
    \item массив просматривается поочередно справа налево и слева направо.
\end{itemize}

\section{Вывод}

В данной работе стоит задача реализации 3 алгоритмов сортировки: быстрая сортировка, сортировка вставками и шейкерная сортировка. Необходимо теоретически оценить трудоемкость этих алгоритмов и проверить все вычисления экспериментально.

