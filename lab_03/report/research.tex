\chapter{Исследовательский раздел}
\label{cha:research}

В данном разделе будет произведено исследование разработанной программы.

\section{Эксперименты по замеру времени}

На рисунках \ref{img:pltsorted}, \ref{img:unsorted}, \ref{img:pltrandom} приведены графики зависимости времени выполнения сортировки от длины сортируемого массива для отсортированных, отсортированных в обратном порядке и произвольных массивов соответственно.

\begin{figure}[H]
    \centering
    \begin{tikzpicture}
        \begin{axis}[
            width=0.8*\linewidth,
            xlabel={Длина массива},
            ylabel={Время (нс)},
            grid=major,
            legend pos=north west,
        ]

        \addplot[color=red]
            table[x=len,y=time,col sep=comma]{./data/qsi.csv};
        \addplot[color=blue]
            table[x=len,y=time,col sep=comma]{./data/ini.csv};
        \addplot[color=green]
            table[x=len,y=time,col sep=comma]{./data/shi.csv};

        \legend{быстрая сортировка, сортировка вставками, шейкерная сортировка}

        \end{axis}
    \end{tikzpicture}
    \caption{Отсортированный массив}
    \label{img:pltsorted}
\end{figure}

\begin{figure}[H]
    \centering
    \begin{tikzpicture}
        \begin{axis}[
            width=0.8*\linewidth,
            xlabel={Длина массива},
            ylabel={Время (нс)},
            grid=major,
            legend pos=north west,
        ]

        \addplot[color=red]
            table[x=len,y=time,col sep=comma]{./data/qsd.csv};
        \addplot[color=blue]
            table[x=len,y=time,col sep=comma]{./data/ind.csv};
        \addplot[color=green]
            table[x=len,y=time,col sep=comma]{./data/shd.csv};

        \legend{быстрая сортировка, сортировка вставками, шейкерная сортировка}

        \end{axis}
    \end{tikzpicture}
    \caption{Обратно отсортированный массив}
    \label{img:unsorted}
\end{figure}

\begin{figure}[H]
    \centering
    \begin{tikzpicture}
        \begin{axis}[
            width=0.8*\linewidth,
            xlabel={Длина массива},
            ylabel={Время (нс)},
            grid=major,
            legend pos=north west,
        ]

        \addplot[color=red]
            table[x=len,y=time,col sep=comma]{./data/qsr.csv};
        \addplot[color=blue]
            table[x=len,y=time,col sep=comma]{./data/inr.csv};
        \addplot[color=green]
            table[x=len,y=time,col sep=comma]{./data/shr.csv};

        \legend{быстрая сортировка, сортировка вставками, шейкерная сортировка}

        \end{axis}
    \end{tikzpicture}
    \caption{Произвольный массив}
    \label{img:pltrandom}
\end{figure}

Замер времени выполнения проводился 10 раз для каждого алгоритма, а в графиках указаны средние значения времени для каждой длины массива.

Данный эксперимент проводился на ноутбуке, подключённом к сети питания. Модель процессора ноутбука: Intel i5-8400H с максимальной тактовой частотой 2.500 ГГц в обычном режиме.

\section{Вывод}
Из полученных графиков видно, что скорость работы быстрой сортировки превосходит две другие в случаях обратно отсортированных и произвольных массивов. При этом шейкерная сортировка уступает сортировке вставками в случаях произвольных и отсортированных массивов.

