\chapter{Аналитический раздел}
\label{cha:analysis}

В данном разделе будет определена теоретическая база, необходимая для реализации поставленных задач.

\section{Описание задачи}

В соответствии с книгой \cite{litr} дадим определение произведения двух матриц. Пусть даны прямоугольные матрицы $A$ и $B$. Размеры этих матриц $n\times{}r$ и $r\times{}m$ соответственно. Тогда результатом умножения матрицы $A$ на матрицу $B$ называется такая матрица $C$ размера $n\times{}m$, что:
\begin{equation}
    c_{ij} = \sum\limits_{k=1}^r(a_{ik}\cdot{}b_{kj})
\end{equation}
где $i = \overline{1, n}$, $j = \overline{1, m}$.

Также важно заметить, что вычисление каждого нового элемента результирующей матрицы не влияет на вычисление следующих, то есть каждый элемент матрицы считается отдельно. Значит, можно произвести распараллеливание вычислений и, тем самым, ускорить их.

\subsection{Алгоритм Винограда}
Данный алгоритм представляет собой альтернативный способ умножения матриц, позволяющий уменьшить количество операций умножения при вычислениях.

В качестве примера рассмотрим два вектора длинной 4: $U = (u_1, u_2, u_3, u_4)$ и $V = (v_1, v_2, v_3, v_4)$. Их скалярное произведение равно:
\begin{equation}
    U\times{}V = u_1\cdot{}v_1 + u_2\cdot{}v_2 + u_3\cdot{}v_3 + u_3\cdot{}v_4
\end{equation}
Левую часть этого равенства можно записать в виде:
\begin{equation}
    (u_1 + v_2)\cdot{}(u_2 + v_1) + (u_3 + v_4)\cdot{}(u_4 + v_3) - u_1\cdot{}u_2 - u_3\cdot{}u_4 - v_1\cdot{}v_2 - v_3\cdot{}v_4
\end{equation}
Это выражение допускает предварительную обработку в случае умножения двух матриц. Для каждой строки можно вычислить выражение:
\begin{equation}
    u_1\cdot{}u_2 + u_3\cdot{}u_4
\end{equation}
А для каждого столбца:
\begin{equation}
    v_1\cdot{}v_2 + v_3\cdot{}v_4
\end{equation}
Таким образом, выходит, что над заранее обработанными данными необходимо выполнить лишь 2 умножения, 5 сложений и 2 вычитания. Данная логика применима и для общего случая умножения матриц.

\section{Вывод}
Умножение матриц необходимый инструмент, для которого есть пути ускорения вычислений за счет уменьшения доли умножения и распараллеливания вычислений.

