\chapter{Исследовательский раздел}
\label{cha:research}

В данном разделе будут производиться эксперементы над корректно реализованной программой.

\section{Эксперименты по замеру времени}

Данный эксперимент состоит в перемножении определенного количества квадратных матриц одинакового размера. Результаты изображены на графиках зависимости времени выполнения от количества матриц.

На рисунке \ref{img:even} приведен график сравнения последовательного умножения матриц и конвейерного. Размер матриц равен 100.

\begin{figure}[H]
    \centering
    \begin{tikzpicture}
        \begin{axis}[
            width=0.8*\linewidth,
            xlabel={Кол-во матриц},
            ylabel={Время (нс)},
            grid=major,
            legend pos=north west,
            xtick={0,100,...,1000}
        ]

        \addplot[color=red]
            table[x=qty,y=time,col sep=comma]{./data/leven.csv};
        \addplot[color=green]
            table[x=qty,y=time,col sep=comma]{./data/ceven.csv};
        \legend{последовательные умножения, конвейерные умножения}

        \end{axis}
    \end{tikzpicture}
    \caption{Чётная длина, сравнение последовательных и конвейерных вычислений}
    \label{img:even}
\end{figure}

На рисунке \ref{img:uneven} приведен график сравнения последовательного умножения матриц и конвейерного. Размер матриц равен 101.

\begin{figure}[H]
    \centering
    \begin{tikzpicture}
        \begin{axis}[
            width=0.8*\linewidth,
            xlabel={Кол-во матриц},
            ylabel={Время (нс)},
            grid=major,
            legend pos=north west,
            xtick={0,100,...,1000}
        ]

        \addplot[color=red]
            table[x=qty,y=time,col sep=comma]{./data/luneven.csv};
        \addplot[color=green]
            table[x=qty,y=time,col sep=comma]{./data/cuneven.csv};
        \legend{непараллелизированный, 1 поток}

        \end{axis}
    \end{tikzpicture}
    \caption{Нечётная длина, сравнение последовательных и конвейерных вычислений}
    \label{img:uneven}
\end{figure}

Эксперимент ставился пять раз, в графиках приведены средние значения времени.

Данный эксперимент проводился на ноутбуке, подключённом к сети питания. Модель процессора ноутбука: Intel i5-8400H с максимальной тактовой частотой 2.500 ГГц в обычном режиме и 8 логическими ядрами.

\section{Вывод}
Исходя из полученных графиков, можно заключить, что конвейерные вычисления дают прирост в скорости по сравнению с последовательными.

